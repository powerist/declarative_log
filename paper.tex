\documentclass{article}

\usepackage{cite}
\usepackage{url}

\begin{document}
\newcommand{\saf}{SAF}


\title{Declarative Data Analytic}
\author{}

\maketitle


\section{Project}
\label{sec:project}

\subsection{Motivation}
\label{sec:motiv}

Existing root cause analysis techniques assume the availability of a known
causality model. However, a network administrator may not have a complete
knowledge of all the causality relationships in the system (Examples?). In
certain cases, the only available information is the logging history of a
system. Therefore, a framework that allows the administrator to easily
create, update, manipulate and query dependency information is of great help. The
framework should also provide a friendly interface that fits the administrator's
high-level understanding of the system. In addition, the framework should also
enable the administrator to easily carry out root cause analysis, given the
dependency information and the logging history.

% \subsection{Goal}
% \label{sec:proj:goal}
% Help the user to define failure models of systems, so that the failure model
% could help identify the root cause of anomalies.

% Develop a declarative language that allows the user to specify/query the root
% cause of a system anomaly, so that the user does not need to handle low-level
% logging data directly.

\section{Meeting Log}
\label{sec:meetlog}

\subsection{5/27}

We discussed about the skeleton of the project. The problem is much clearer
after the meeting. The goal of the project, at this stage, is to develop a
declarative language that helps a network administrator to specify/query the root cause of a
system problem, given overwhelming log information.

\subsection{6/3}
\label{sec:meeting:63}

\paragraph{Progress}
\begin{itemize}
\item Read the paper ``A survey of fault localization techniques in computer
networks'': They summarized a number of techniques that find the root cause of a
system error with a given causality graph. 

\item Read the paper ``Efficient Querying and Maintenance of Network Provenance at
Internet-Scale'': They proposed a graph-based data structure to maintain network
provenance, i.e., dependency relationship. Also, different forms of
representation for specifying the causal relationship were given, such as
algebraic representation and BDD.

\item Scan through the NEC network data.
\end{itemize}

\paragraph{Meeting summary.}
In the meeting, we further clarified the problem.

In the current design, our framework takes two inputs:
(1) a set of logs from the network (format undefined); and
(2) a detected anomaly (format undefined).

The output of the framework is the root cause that could explain the detected
anomaly. The format of the root cause is also unknown now (could be an event or
a set of events). 

Within the framework, we have a library which stores behavior
models (i.e., causal relationship). There will also be a root cause analysis
engine that is responsible
for generating the root cause based on the anomaly, the logs and the
library. A possible design for the root cause analysis engine
could be first to find the behavior models that have the anomaly as root---These
models are called failure models for the anomaly---then the
engine finds all log sequences satisfying the behavior model (we can
leverage \saf{} for this part), and runs the root cause analysis algorithm on
the log sequences found. 

The behavior library needs to be specified by the user---it is almost impossible
to infer the behavior models without human input---and a language is needed for
the user to specify, manipulate, and query the library. The language could be any
existing language, such as \saf{} and Datalog, or a newly designed one.

\paragraph{To-do}
\begin{itemize}
\item More survey on the root cause analysis is needed. The focus would be on
  (1) What's the current status of root cause analysis?; (2) whether anyone has
  tried to manage causality as a database? If so, how did they do it?
\item The real data is worth further exploration. Hopefully we can have more
  insight from the data.
\end{itemize}


\section{Reading List}
\label{sec:read}

\begin{itemize}

\item Troubleshooting Blackbox SDN Control Software with Minimal Causal
Sequences. (link: \url{http://www.eecs.berkeley.edu/~rcs/research/sts.pdf})

\item Fault Management in Distributed Systems. (link:
\url{http://repository.upenn.edu/cgi/viewcontent.cgi?article=1960&context=cis_reports}) 

\item A survey of fault localization techniques in computer networks. (link: \url{http://www.eecis.udel.edu/~sethi/papers/04/socp04.pdf})

\end{itemize}

\section{\saf{}}
\label{sec:saf}

\subsection{Paper}
\label{sec:safpaper}

\paragraph{Overview.}
The purpose of \saf{}~\cite{arun2011} is to bridge the semantic gap between raw
data and the user's high-level view of the system. \saf{} achieves the goal by
introducing {\em behavior model} --- a logic-based abstraction that allows the user to
concisely specify desired relationship between different events of the system --- along
with a low-level implementation that maps user-defined behaviors (i.e., logic
formulas) to satisfying log traces. 

\paragraph{Implementation.}
The implementation is composed of five components: 
(1) knowledge base, which provides domain-specific knowledge pre-specified as behavior
models; 
(2) data normalizer, which normalizes data from different sources into unified
events --- an internal data model for storage and query.
(3) event storage, an SQL database storing events.
(4) analysis engine, which takes a user-specified behavior model as input, and
extracts the satisfying set of events (or event sequences) based on defined
semantics.
(5) presentation engine, an output system.

\paragraph{Comments.}
\begin{itemize}
\item Is the specification for exact time comparison useful? Due to possible
  clock shift between different devices in a distributed system, the timestamps
  in logs extracted from different devices may not be comparable at all. I
  believe relative time ordering makes more sense in a network.
\end{itemize}


\subsection{\saf{} Tool}
\label{sec:saftool}

\paragraph{Progress.}

\begin{itemize}
\item Tutorial finished: I have tried the example TCP records on the \saf
  website, and generated the traces that satisfy the three-way-handshake
  behavior.
\item Trying \saf{} on real-data now... (Detailed information of the data is in
  the email).
\end{itemize}

\paragraph{Questions:}


\subsection{A root cause localization model for large scale systems}
\label{sec:rootcause}

The paper (bib not found online) presents a model capturing essential features of the root cause
localization process. In the model, a system is modeled as a set of components
that interact with each other. Observable behaviors of the system are modeled as
quarks --- the smallest end-to-end observable unit of specific service. Each
quark is composed of:
(1) the set of components used by the quark (could be implemented as a list); and
(2) a health result, signifying the failure or the success of the quark. (could
be implemented as a boolean variable).

Partial failure is modeled with probability: Given a component $C_i$,
probability $p_i$ represents the failure probability of a quark that utilizes
component $C_i$.

The paper further defines the problem of root cause localization, which has two
versions: deterministic version and statistical version. Only the deterministic
version is relevant to our project: Given several quarks of the system, some of
which succeed and others fail, estimate the set of components that could have partial
failure i.e., list of components $C_i$ with $p_i > 0$.

\paragraph{Comments}

\begin{itemize}
\item Is failure always modeled using probability? What are other possible
  failure models?
\end{itemize}



\bibliography{paper}{}
\bibliographystyle{plain}

\end{document}
%%% Local Variables:
%%% mode: latex
%%% TeX-master: t
%%% End:
