\documentclass{article}

\usepackage{cite}

\begin{document}
\newcommand{\saf}{SAF}


\title{Declarative Data Analytic}
\author{}

\maketitle

\section{Summary for \saf}
\label{sec:saf}

\subsection{Paper}
\label{sec:safpaper}

\paragraph{Overview.}
The purpose of \saf\cite{arun2011} is to bridge the semantic gap between raw
data and the user's high-level view of the system. \saf achieves the goal by
introducing {\em behavior model} --- a logic-based abstraction that allows the user to
concisely specify desired relationship between different events of the system --- along
with a low-level implementation that maps user-defined behaviors (i.e., logic
formulas) to satisfying log traces. 

\paragraph{Implementation.}
The implementation is composed of five components: 
(1) knowledge base, which provides domain-specific knowledge pre-specified as behavior
models; 
(2) data normalizer, which normalizes data from different sources into unified
events --- an internal data model for storage and query.
(3) event storage, an SQL database storing events.
(4) analysis engine, which takes a user-specified behavior model as input, and
extracts the satisfying set of events (or event sequences) based on defined
semantics.
(5) presentation engine, an output system.

\paragraph{Comments.}
\begin{itemize}
\item Is the specification for exact time comparison useful? Due to possible
  clock shift between different devices in a distributed system, the timestamps
  in logs extracted from different devices may not be comparable at all. I
  believe relative time ordering makes more sense in a network.

\paragraph{Questions.}
\begin{itemize}
\item What
\end{itemize}
  
\end{itemize}


\subsection{\saf tool}
\label{sec:saftool}



\bibliography{paper}{}
\bibliographystyle{plain}

\end{document}
%%% Local Variables:
%%% mode: latex
%%% TeX-master: t
%%% End:
