\documentclass{article}

\usepackage{cite}

\begin{document}
\newcommand{\saf~}{SAF}

%%% Local Variables:
%%% mode: latex
%%% TeX-master: "paper"
%%% End:


\title{Declarative Data Analytic}
\author{}

\maketitle

\section{\saf}
\label{sec:saf}

\subsection{Paper}
\label{sec:safpaper}

\paragraph{Overview.}
The purpose of \saf\cite{arun2011} is to bridge the semantic gap between raw
data and the user's high-level view of the system. \saf achieves the goal by
introducing {\em behavior model} --- a logic-based abstraction that allows the user to
concisely specify desired relationship between different events of the system --- along
with a low-level implementation that maps user-defined behaviors (i.e., logic
formulas) to satisfying log traces. 

\paragraph{Implementation.}
The implementation is composed of five components: 
(1) knowledge base, which provides domain-specific knowledge pre-specified as behavior
models; 
(2) data normalizer, which normalizes data from different sources into unified
events --- an internal data model for storage and query.
(3) event storage, an SQL database storing events.
(4) analysis engine, which takes a user-specified behavior model as input, and
extracts the satisfying set of events (or event sequences) based on defined
semantics.
(5) presentation engine, an output system.

\paragraph{Comments.}
\begin{itemize}
\item Is the specification for exact time comparison useful? Due to possible
  clock shift between different devices in a distributed system, the timestamps
  in logs extracted from different devices may not be comparable at all. I
  believe relative time ordering makes more sense in a network.
\end{itemize}


\subsection{\saf tool}
\label{sec:saftool}

\paragraph{Progress.}

\begin{itemize}
\item Tutorial finished: I have tried the example TCP records on the \saf
  website, and generated the traces that satisfy the three-way-handshake
  behavior.
\item Trying \saf on real-data now...
\end{itemize}

\paragraph{Questions:}

\begin{itemize}
\item Are GISP\_Original, Thin\_client\_environment and Tamagawa related data? 
\item Do all of them contain anomalies? 
\item What has been done to diagnose/solve the anomalies, if any?

\end{itemize}

\subsection{A root cause localization model for large scale systems}
\label{sec:rootcause}

The paper presents a model capturing essential features of the root cause
localization process. In the model, a system is modeled as a set of components
that interact with each other. Observable behaviors of the system are modeled as
quarks --- the smallest end-to-end observable unit of specific service. Each
quark is composed of:
(1) the set of components used by the quark (could be implemented as a list); and
(2) a health result, signifying the failure or the success of the quark. (could
be implemented as a boolean variable).

Partial failure is modeled with probability: Given a component $C_i$,
probability $p_i$ represents the failure probability of a quark that utilizes
component $C_i$.

The paper further defines the problem of root cause localization, which has two
versions: deterministic version and statistical version. Only the deterministic
version is relevant to our project: Given several quarks of the system, some of
which succeed and others fail, estimate the set of components that could have partial
failure i.e., list of components $C_i$ with $p_i > 0$.

\paragraph{Comments}

\begin{itemize}
\item Is failure always modeled using probability? What are other possible
  failure models?
\end{itemize}

\section{Project}
\label{sec:project}

\subsection{Goal}
\label{sec:proj:goal}
Help the user to define failure models of systems, so that the failure model
could help identify the root cause of anomalies.

\bibliography{paper}{}
\bibliographystyle{plain}

\end{document}
%%% Local Variables:
%%% mode: latex
%%% TeX-master: t
%%% End:
